\documentclass[12pt]{report}
\usepackage{amsmath, amssymb, amsfonts, amsthm, mathtools,mathrsfs}
\usepackage{thmtools}
\usepackage[utf8]{inputenc}
\usepackage[inline]{enumitem}
\usepackage[colorlinks=true]{hyperref}
\usepackage{multicol}
\usepackage{witharrows}
\usepackage{tikz}
\usetikzlibrary{automata,positioning}
\usetikzlibrary{decorations.markings}
\usepackage{verbatim}
\usetikzlibrary{arrows.meta} 
\usepackage{witharrows}
\usepackage[useregional, showdow]{datetime2}
\usepackage{physics}
\DTMlangsetup[en-GB]{abbr}
\usepackage{xcolor}
\usepackage[normalem]{ulem}
\usetikzlibrary{chains,shapes.multipart}
\usepackage{algorithm}
\usepackage{algpseudocode}

\usepackage{bbm}
\usepackage[page,toc,titletoc,title]{appendix}
\usepackage{tocloft}
\setlength\parindent{0pt}
\usepackage{parskip}

\def\D{\mathrm{d}}
\def\I{\mathbb{I}}
\def\P{\mathbb{P}}
\def\E{\mathbb{E}}
\def\Var{\text{Var}}
\def\F{\mathcal{F}} 
\def\ccdf{\overline{F}} 
\newcommand*{\thead}[1]{\multicolumn{1}{c}{\bfseries #1}}
\renewcommand{\arraystretch}{1}
\newcommand{\indep}{\perp \!\!\! \perp}

\usepackage[framemethod=tikz]{mdframed}
\mdfdefinestyle{theoremstyle}{%
	% linecolor=gray,linewidth=1pt,%
	% frametitlerule=true,%
	frametitlebackgroundcolor=white,
	% backgroundcolor=  gray!20,	
	bottomline=false, topline=false, rightline=false, leftline=true,
	innerlinewidth=0.7pt, outerlinewidth=0.2pt, middlelinewidth=2pt, middlelinecolor=white, %
	innerleftmargin=6pt,
	% innertopmargin=-1pt,
	skipabove=10pt,
	% fontcolor=blue,
	% innerbottommargin=-0.5pt,
}
\mdtheorem[style=theoremstyle]{defn}[thm]{Definition}[section]
\mdtheorem[style=theoremstyle]{lem}[thm]{Lemma}
\mdtheorem[style=theoremstyle]{prop}[thm]{Proposition}
\mdtheorem[style=theoremstyle]{thm}{Theorem}[section]
\mdtheorem[style=theoremstyle]{cor}{Corollary}[section]


\newcommand*{\doublerule}{\hrule width \hsize height 1pt \kern 0.5mm \hrule width \hsize height 2pt}
\newcommand{\doublerulefill}{\leavevmode\leaders\vbox{\hrule width .1pt\kern1pt\hrule}\hfill\kern0pt}
\def\ddfrac#1#2{\displaystyle\frac{\displaystyle #1}{\displaystyle #2}}


%\newcommand{\Res}{\operatorname{Res}}

\theoremstyle{definition}
% \numberwithin{thm}{section}
% \newtheorem{lem}[thm]{Lemma}
% \newtheorem{defn}[thm]{Definition}
% \newtheorem{prop}[thm]{Proposition}
% \newtheorem{cor}[thm]{Corollary}
% \newtheorem{ex}{Example}


\let\emptyset\varnothing

\usepackage{titlesec}
\titleformat{\section}[block]{\Large\filcenter\bfseries}{\S\thesection.}{0.25cm}{\Large}
\titleformat{\subsection}[block]{\large\bfseries\sffamily}{\S\S\thesubsection.}{0.2cm}{\large}

\usepackage[a4paper]{geometry}
\usepackage{lipsum}
\usepackage{xcolor,cancel}

\usepackage{cleveref}
\crefname{thm}{Theorem}{Theorems}
\crefname{lem}{Lemma}{Lemmas}
\crefname{defn}{Definition}{Definitions}
\crefname{prop}{Proposition}{Propositions}
\crefname{cor}{Corollary}{Corollaries}
\crefname{equation}{}{}
\DeclareMathOperator*{\argmin}{arg\,min}

\usepackage{mdframed}
\newenvironment{blockquote}
{\begin{mdframed}[skipabove=0pt, skipbelow=0pt, innertopmargin=4pt, innerbottommargin=4pt, bottomline=false,topline=false,rightline=false, linewidth=2pt]}
{\end{mdframed}}
\newenvironment{soln}{\begin{proof}[Solution]}{\end{proof}}

\title{Stochastic Optimisation\\}
\author{Ishan Kapnadak}
\date{Spring Semester 2020-21\\~\\Updated on: \textcolor{blue}{\DTMToday}}

\begin{document}
\tikzset{lab dis/.store in=\LabDis,
  lab dis=-0.4,
  ->-/.style args={at #1 with label #2}{decoration={
    markings,
    mark=at position #1 with {\arrow{>}; \node at (0,\LabDis) {#2};}},postaction={decorate}},
  -<-/.style args={at #1 with label #2}{decoration={
    markings,
    mark=at position #1 with {\arrow{<}; \node at (0,\LabDis)
    {#2};}},postaction={decorate}},
  -*-/.style args={at #1 with label #2}{decoration={
    markings,
    mark=at position #1 with {{\fill (0,0) circle (1.5pt);} \node at (0,\LabDis)
    {#2};}},postaction={decorate}},
  }
\maketitle

\begin{abstract}
    \begin{center}
        Lecture Notes for the course EE 736 : Stochastic Optimisation taught in Spring 2022 by Prof. Vivek Borkar.
    \end{center}
\end{abstract}

\tableofcontents
\newpage
\chapter{Stochastic Approximation}
\section{The Robbins-Monro Algorithm}

The basic problem we consider is to solve $h(\mathbf{x}) = 0$ given noisy measurements of $h$. That is, we are given access to a black box that, on input $\mathbf{x} \in \mathbb{R}^d$, gives as output $h(\mathbf{x}) + \text{noise}$. To this end, we have the \textit{Robbins-Monro algorithm}.

\medskip

\textbf{Robbins-Monro Algorithm.} Starting with $\mathbf{x}_0 \in \mathbb{R}^d$, do: 
\[  
    \mathbf{x}(n+1) \vcentcolon= \mathbf{x}(n) + a(n) \left[ h(\mathbf{x}(n)) + M(n+1) \right], \quad n \geq 0.
\]  
Here, the (non-negative) stepsize sequence (or learning parameter) $\{a(n)\}$ satisfies
\[
    \sum_{n} a(n) = \infty \quad \text{and} \quad \sum_{n} a(n)^2 < \infty. 
\]

A typical example of such a stepsize sequence is $\frac{1}{n}, \frac{1}{n\log n}, \frac{1}{n^{2/3}}$, and so on. Further, we make the following assumptions. 

\begin{enumerate}
    \item $h \colon \mathbb{R}^d \to \mathbb{R}^d$ is Lipschitz, that is, $\exists L \geq 0$ such that
    \[
        \norm{h(\mathbf{x}) - h(\mathbf{y})} \leq L\norm{\mathbf{x} - \mathbf{y}}
    \]
    for all $\mathbf{x,y} \in \mathbb{R}^d$. 
    
    \item $\{M(n)\}$ is a square integrable martingale difference sequence. That is, for 
    \[
        \mathcal{F}_n \vcentcolon= \sigma \left( \mathbf{x}_0, M_m, m \leq n \right), n \geq 0
    \]
    we have
    \[
        \E \left[ \norm{M(n)}^2 \right] < \infty
    \]
    and in addition, we have that it is uncorrelated with the past. That is, 
    \[
        \E \left[ M_i(n+1) \mid \mathcal{F}_n \right] = 0 \quad \forall i.
    \]
    Furthermore, we assume that for some $K > 0$, 
    \[
        \E \left[ \norm{M(n+1)}^2 \mid \mathcal{F}_n \right] \leq K(1 + \norm{\mathbf{x}(n)}^2) \quad \forall n \geq 0.
    \]
    In particular, if 
    \[
        \sup_n \norm{\mathbf{x}(n)} < \infty \quad \text{a.s.}, 
    \]
    then,
    \[
        \sup_n \E \left[ \norm{M(n+1)}^2 \mid \mathcal{F}_n \right] < \infty \quad \text{a.s.}.
    \]
\end{enumerate}

This algorithm is actually more general than it appears. Suppose the algorithm is 
\[
    \mathbf{x}(n+1) \vcentcolon= \mathbf{x}(n) + a(n) f\left( \mathbf{x}(n), \xi(n+1) \right), \quad n \geq 0
\]
where $\{\xi(n)\}$ are i.i.d. This is often how most recursive algorithms are stated. The above algorithm can be put into the form of Robbins-Monro algorithm by choosing
\begin{align*}
    h(\mathbf{x}) &\vcentcolon= \E \left[ f(\mathbf{x}, \xi(n)) \right] \\
    &= \E \left[ f(\mathbf{x}(n), \xi(n+1)) \mid \mathbf{x}(n) = \mathbf{x} \right] \\
    &= \E \left[ f(\mathbf{x}(n), \xi(n+1)) \mid \mathcal{F}_n \right]
\end{align*}
and
\[
    M(n+1) \vcentcolon= f(\mathbf{x}(n),\xi(n+1)) - h(\mathbf{x}(n)).
\]

A common example of Robbins-Monro algorithm is stochastic gradient descent, where we set $h = -\nabla f$. Robbins-Monro algorithm also finds uses in many reinforcement learning algorithms. Some advantages of the Robbins-Monro algorithm are listed as follows. 

\begin{enumerate}
    \item It typically requires a small amount of computation and memory per iterate.
    \item It is incremental in nature, that is, it makes only a small change in the current iterate at each step.
    \item The slowly decreasing stepsize $\{a(n)\}$ captures the exploration-exploitation trade-off.
    \item It averages out the noise, which can be thought of as a generalisation of the Strong Law of Large Numbers.
\end{enumerate}

Another common approach to solving the same problem is the ODE (Ordinary Differential Equation) approach, which treats the iterate as a noisy discretisation of the ODE 
\[
    \dot{\mathbf{x}}(t) = h(\mathbf{x}(t)).
\]
Recall the Euler scheme for solving this ODE:
\[
    \mathbf{x}(n+1) \vcentcolon= \mathbf{x}(n) + ah(\mathbf{x}(n)), \quad n \geq 0,
\]
where $a > 0$ is a small discrete time step. Thus the Robbins-Monro algorithm can be viewed as a Euler scheme to approximate the ODE with slowly decreasing time steps $\{a(n)\}$ and measurement noise. With this in mind, we have the following interpretation of the Robbins-Monro conditions on the step size $\{a(n)\}$. 

\begin{enumerate}
    \item $\sum_n a(n) = \infty$ ensures that the entire time axis is covered. This is essential because we want to track the asymptotic behaviour of the ODE. 
    \item $\sum_n a(n)^2 < \infty$ ensures that the approximation of the ODE gets better with time. In particular, $a(n) \to 0$ ensures that errors due to discretisation are asymptotically zero, and $\sum_n a(n)^2 < \infty$ ensures that errors due to the martingale difference noise are asymptotically zero almost surely, since multiplication by $a(n)$ reduces the conditional variance of the noise. 
\end{enumerate}

As an example, consider an initially empty urn to which one ball, either red or blue, is added at each time step. Let 
\[
    \xi(n) \vcentcolon= \I\{n^{\text{th}}\text{ ball is red}\} = \begin{cases}
        1 & \text{if the $n^{\text{th}}$ ball is red, and} \\
        0 & \text{otherwise.}
    \end{cases}
\]  
Let $S(n) \vcentcolon= \sum_{m=1}^n \xi(m)$ be the total number of red balls at time $n$, and let $x(n) \vcentcolon= \frac{S(n)}{n}$ be the fraction of red balls at time $n$. Then, we have
\begin{align*}
    x(n+1) &= \frac{1}{n+1} \sum_{m=1}^{n+1} \xi(m) \\
    &= \frac{1}{n+1} \sum_{m=1}^{n} \xi(m) + \frac{\xi(n+1)}{n+1} \\
    &= \left( \frac{n}{n+1} \right) \frac{\sum_{m=1}^n \xi(m)}{n} + \frac{\xi(n+1)}{n+1} \\
    &= \left( 1 - \frac{1}{n+1} \right) x(n) + \frac{\xi(n+1)}{n+1} \\
    &= x(n) + a(n) ( \xi(n+1) - x(n))
\end{align*}
for $a(n) \vcentcolon= \frac{1}{n+1}$ which satisfies the Robbins-Monro conditions. Now, suppose that
\[
    \P \left( \xi(n+1) = 1 \mid \xi(m), m \leq n \right) = p(x(n))
\]
for some continuously differentiable function $p \colon [0,1] \to [0,1]$. Then, we have
\begin{align*}
    x(n+1) &= x(n) + a(n) ( \xi(n+1) - x(n)) \\
    &= x(n) + a(n) \left[ (p(x(n)) - x(n)) + (\xi(n+1) - p(x(n)))\right] \\
    &= x(n) + a(n) \left[ h(x(n)) + M(n+1) \right]
\end{align*}
for $h(x) \vcentcolon= p(x) - x$, and $M(n+1) \vcentcolon= \xi(n+1) - p(x(n))$. Since $\E \left[ \xi(n+1) \mid \xi(m), m \leq n \right] = p(x(n))$ for all $n$, we have that $\{M(n)\}$ is a martingale difference sequence. Since $\abs{M(n)} \leq 2$, the bound on conditional second moment is free. The limiting ODE is
\[
    \dot{x}(t) = p(x(t)) - x(t).
\]
Under our hypothesis of continuous differentiability of $p$, this has a unique solution for any initial condition. Set $x(0) = x_0 \in [0,1]$. We have $p(0) - 0 \geq 0$, and $p(1) - 1 \leq 0$. Since $x(t) \in [0,1]$ for all $t \geq 0$, $x(t)$ must converge to a point in $[0,1]$. If at $x_0$, we have that $p(x_0) = x_0$, then we are already at equilibrium. If not, suppose that $p(x_0) > x_0$, then $x(t)$ is increasing but bounded by $1$, so it must converge. A similar argument works for $p(x_0) < x_0$. But does an equilibrium exist? The answer is yes. Since $p(0) - 0 \geq 0$, and $p(1) - 1 \leq 0$, we have by continuity that there exists $x \in [0,1]$ such that $p(x) = x$. In fact, there can be more than one equilibria. An equilibrium $x^*$ satisfies $p(x^*) = x^*$ and is stable if $p^{\prime}(x^*) < 1$ and unstable if $p^{\prime}(x^*) > 1$. Under some additional technicalities, we can show that $x(t)$ converges to one of the stable equilibria almost surely, and the probability of convergence to any stable equilibrium is strictly positive. 

\section{Ordinary Differential Equations}

We consider the ODE in $\mathbb{R}^d$, $d \geq 1$, given by
\[
    \dot{\mathbf{x}}(t) = h(\mathbf{x}(t)), \quad \mathbf{x}(0) = \mathbf{x}_0.
\]
A problem is said to be \emph{well-posed} if 
\begin{enumerate}
    \item it has a solution, 
    \item the solution is unique, and
    \item the solution depends continuously on problem parameters. 
\end{enumerate}

For ODEs, this translates to the ODE having a unique solution for all time that depends continuously on the initial condition.

\medskip

We say that $h \colon \mathbb{R}^d \to \mathbb{R}^d$ satisfies a (global) Lipschitz condition if for some $L > 0$, we have
\[
    \norm{h(\mathbf{x}) - h(\mathbf{y})} \leq L \norm{\mathbf{x} - \mathbf{y}} \quad \forall \mathbf{x},\mathbf{y} \in \mathbb{R}^d. 
\]
$h$ is locally Lipschitz if for all $R > 0$, there exists an $L_R > 0$ such that
\[
    \norm{h(\mathbf{x}) - h(\mathbf{y})} \leq L_R \norm{\mathbf{x} - \mathbf{y}} \quad \forall \mathbf{x},\mathbf{y} \in \mathcal{B}_R \vcentcolon= \left\{ \mathbf{z} \in \mathbb{R}^d \mid \norm{\mathbf{z}} \leq R  \right\}
\]

\begin{lem}[Gronwall's Inequality]
    Suppose $0 \leq y \colon [0,T] \colon \mathbb{R}$ is differentiable and satisfies
    \[
        y(t) \leq C + K \int_0^t y(s) \, \D s, \quad t \in [0,T]
    \]
    for some $C,K > 0$. Then, 
    \[
        y(t) \leq Ce^{Kt}, \quad t \in [0,T].
    \]
\end{lem}

\begin{proof}
    Let $z(t) \vcentcolon= \int_0^t y(s) \, \D s$, $t \geq 0$. Then, 
    \begin{align*}
        &\dot{z}(t) = y(t) \leq C + Kz(t) \\
        \implies &e^{-Kt}(\dot{z}(t) - Kz(t)) \leq Ce^{-Kt} \\
        \implies & \frac{\D}{\D t} \left( e^{-Kt} z(t) \right) \leq Ce^{-Kt}, \, z(0) = 0.
    \end{align*}
    Integrating both sides from $0$ to $t$, we get
    \begin{align*}
        &e^{-Kt} z(t) \leq \frac{C}{K} (1 - e^{-Kt}) \\
        \implies &z(t) \leq \frac{C}{K} (e^{Kt} - 1)
    \end{align*}
    Now, we have
    \begin{align*}
        &y(t) \leq C + Kz(t) \leq C + C(e^{Kt} - 1) \\
        \implies & y(t) \leq Ce^{Kt}. \qedhere
    \end{align*}
\end{proof}

\begin{thm}
    If $h$ is Lipschitz, then the ODE $\left\{ \dot{\mathbf{x}}(t) = h(\mathbf{x}(t)), \mathbf{x}(0) = \hat{\mathbf{x}} \right\}$ is well-posed. 
\end{thm}
\begin{proof}
    We first show existence. Fix $T \in (0, 1/L)$ and a continuous function $\mathbf{x}_0 \colon [0,T] \to \mathbb{R}^d$ with $\mathbf{x}_0(0) = \hat{\mathbf{x}}$. Recursively define
    \begin{equation*}
        \mathbf{x}_{n+1}(t) \vcentcolon= \hat{\mathbf{x}} + \int_0^t h(\mathbf{x}_n(s)) \, \D s, \quad t \in [0,T]. \tag{\(\dagger\)}
    \end{equation*}
    These are called \emph{Picard iterations}. Then for $n \geq 1$, we have
    \begin{align*}
        \norm{\mathbf{x}_{n+1}(t) - \mathbf{x}_n(t)} &= \norm{\int_0^t \left( h(\mathbf{x}_n(s)) - h(\mathbf{x}_{n-1}(s)) \, \right)\D s} \\
        &\leq \int_0^t \norm{h(\mathbf{x}_n(s)) - h(\mathbf{x}_{n-1}(s))} \, \D s \\ 
        &\leq L\int_0^t \norm{\mathbf{x}_n(s) - \mathbf{x}_{n-1}(s)} \, \D s \\
        &\leq LT \max_{s \in [0,T]} \norm{\mathbf{x}_n(s) - \mathbf{x}_{n-1}(s)}
    \end{align*}
    
    Thus, 
    \[
        \max_{t \in [0,T]} \norm{\mathbf{x}_{n+1}(t) - \mathbf{x}_n(t)} \leq LT \max_{t \in [0,T]} \norm{\mathbf{x}_n(t) - \mathbf{x}_{n-1}(t)}
    \]
    
    Applying this repeatedly, we get
    \begin{align*}
        \max_{t \in [0,T]} \norm{\mathbf{x}_{n+1}(t) - \mathbf{x}_n(t)} &\leq (LT)^n \max_{t \in [0,T]} \norm{\mathbf{x}_1(t) - \mathbf{x}_0(t)} \text{ for } n \geq 0 \\
        &\implies \sum_{n=0}^{\infty} \max_{t \in [0,T]}\norm{\mathbf{x}_{n+1}(t) - \mathbf{x}_n(t)} < \infty. 
    \end{align*}
    
    Thus, $\mathbf{x}_n(t) = \mathbf{x}_0(t) + \sum_{m=0}^{n-1} (\mathbf{x}_{m+1}(t) - \mathbf{x}_m(t))$ converges to some $\mathbf{x}(t)$ uniformly in $t \in [0,T]$. Passing to the limit as $n \uparrow \infty$ in $(\dagger)$, we have 
    \[
        \mathbf{x}(t) \vcentcolon= \hat{\mathbf{x}} + \int_0^t h(\mathbf{x}(s)) \, \D s, \quad t \in [0,T]
    \]
    Thus, $\mathbf{x}$ satisfies the ODE with $\mathbf{x}(0) = \hat{\mathbf{x}}$. We repeat the above procedure for $[T,2T], [2T,3T]$, and so on.
    
    \medskip
    
    We now prove uniqueness. Consider $\dot{\mathbf{x}}(t) = h(\mathbf{x}(t))$, $\dot{\mathbf{y}}(t) = h(\mathbf{y}(t))$, $t \geq 0$ with $\mathbf{x}(0) = \mathbf{y}(0)$. Then,
    \[
        \norm{\mathbf{x}(t) - \mathbf{y}(t)} \leq L \int_0^t \norm{\mathbf{x}(s) - \mathbf{y}(s)} \, \D s \implies \norm{\mathbf{x}(t) - \mathbf{y}(t)} = 0 \quad \forall t \geq 0,
    \]
    where the last implication follows from Gronwall's inequality. This concludes uniqueness. In general, for $\mathbf{x}(0) = \hat{\mathbf{x}}$ and $\mathbf{y}(0) = \hat{\mathbf{y}}$, we have 
    \begin{align*}
        &\norm{\mathbf{x}(t) - \mathbf{y}(t)} \leq \norm{\hat{\mathbf{x}} - \hat{\mathbf{y}}} + L \int_0^t \norm{\mathbf{x}(s) - \mathbf{y}(s)} \, \D s \\
        &\implies \norm{\mathbf{x}(t) - \mathbf{y}(t)} \leq e^{Lt} \norm{\hat{\mathbf{x}} - \hat{\mathbf{y}}},
    \end{align*}
    by the Gronwall's inequality implying continuous dependence on the initial condition. Hence, the ODE is well-posed.
\end{proof}

A few remarks:

\begin{enumerate}
    \item Picard iteration is not a good computational scheme. In practice, Euler scheme is the most basic choice. Suppose $h$ is bounded and let $a \vcentcolon= \frac{T}{N}$ where $N \gg 1$. Now, let
    \[
        \mathbf{X}_N((n+1)a) \vcentcolon= \mathbf{X}_N(na) + ah(\mathbf{X}_N(na)), \quad 0 \leq n < N.
    \]
    We interpolate linearly to get
    \[
        \mathbf{X}_N(t) \vcentcolon= \mathbf{X}_N(na) + (t - na) h(\mathbf{X}_N(na)), \quad t \in [na, (n+1)a].
    \]
    Then, as $N \uparrow \infty$, $\mathbf{X}_N(t), t \in [0,T]$ converges to a solution of the ODE uniformly on $[0,T]$. This too proves the existence of a solution and needs only the continuity of $h$. However, uniqueness may fail. In numerical analysis, more sophisticated discretisations are available. 
    
    \item A local Lipschitz condition on $h$ gives local well-posedness for a small time interval, but the solution may not exist for all time.
    
    \item The linear growth condition shown below suffices for a solution to exist for all time:
    \[
        \norm{h(\mathbf{x}} \leq K(1 + \norm{\mathbf{x}})
    \]  
    for some $K > 0.$ Then, we have
    \begin{align*}
        \norm{\mathbf{x}(t)} &\leq \norm{\mathbf{x}(0)} + \norm{\int_0^t h(\mathbf{x}(s) \, \D s} \leq \norm{\mathbf{x}(0)} + \int_0^t K(1 + \norm{\mathbf{x}(s)}) \, \D s \\
        &\implies \norm{\mathbf{x}(t)} \leq (\norm{\mathbf{x}(0)} + KT) e^{Kt}, \quad t \in [0,T],
    \end{align*}
    by Gronwall's inequality. We further note that the Lipschitz condition implies linear growth. A proof of this is left as an exercise for the reader. 
    
    \item A symmetric well-posedness theory can be developed for $t \leq 0$. Thus, for Lipschitz $h$, there is a unique solution for all $t \in \mathbb{R}$.
    
    \item We also have a \emph{discrete} Gronwall's inequality, which is proved similarly. Let $x_n \geq 0, a_n \geq 0, n \geq 0$, and $C,K > 0$ such that
    \[
        x_{n+1} \leq C + K \sum_{m=0}^n a_mx_m \quad \forall n \geq 0.
    \]  
    Then, $x_{n+1} \leq C e^{K \sum_{m=0}^n a_m}$ for all $n \geq 0$.
\end{enumerate}

We now take a more qualitative look at ODEs. Assume well-posedness. There are two broad ways of thinking about ODEs. 

\begin{enumerate}
    \item We can think of the ODE as the graph of $t \mapsto \mathbf{x}(t) \in \mathbb{R}^d$, that is, we think of $\mathbf{x}(t)$ as a function of time. The component-wise time derivative at $t$ is $h(\mathbf{x}(t))$. 
    \item We can think of the ODE as a trajectory, or a curve $\mathbf{x}(\cdot)$ in $\mathbb{R}^d$ with $t$ as a running parameter. This is also called a phase portrait. The tangent at point $\mathbf{x}$ on the curve is $h(\mathbf{x})$. One often flips this picture around and imagines a vector $h(\mathbf{x})$ at each point $\mathbf{x}$ (a vector field) and think of trajectories as curves drawn that are tangent to the vector field at all points (integral curves).
\end{enumerate}

\begin{defn}[Limit Sets]
The $\omega$-limit set of a trajectory $\mathbf{x}(\cdot)$ is the set of all points $\mathbf{x}$ such that $\exists \, t_n\\uparrow \infty$ such that $\mathbf{x}(t_n) \to \mathbf{x}$, that is, the set of limit points of $\mathbf{x}(t)$ as $t \uparrow \infty$. One can show that this set is closed but can be empty. The $\alpha$-limit set is defined similarly for $t_n \downarrow -\infty$.
\end{defn} 

\begin{defn}[Invariance]
A set $A \subseteq \mathbb{R}^d$ is said to be \emph{positively invariant} if $\mathbf{x}(0) \in A \implies \mathbf{x}(t) \in A \quad \forall t \geq 0$. Negative invariance is defined similarly. A set that is both positively and negatively invariant is said to be \emph{invariant}.
\end{defn} 

\begin{prop}
    The $\omega$- and $\alpha$-limit sets are invariant.
\end{prop}

\begin{defn}[Liapunov Stable]
    An equilibrium $\mathbf{x}^*$ is said to be \emph{Liapunov stable} if given $\epsilon > 0$, there exists a $\delta > 0$ such that
    \[
        \norm{\mathbf{x}(0) - \mathbf{x}^*} < \delta \implies \norm{\mathbf{x}(t) - \mathbf{x}^*} < \delta \quad \forall t \geq 0.
    \]
\end{defn}
\begin{defn}[Asymptotically Stable]
    An equilibrium $\mathbf{x}^*$ is said to be \emph{asymptotically stable} if it is Liapunov stable and there exists an open neighbourhood $\mathcal{O}$ of $\mathbf{x}^*$ such that 
    \[
        \mathbf{x}(0) \in \mathcal{O} \implies \mathbf{x}(t) \to \mathbf{x}^*.
    \]
\end{defn}
\begin{defn}[Domain of Attraction]
    The largest positively invariant open set $\mathcal{D}$ such that 
    \[
        \mathbf{x}(0) \in \mathcal{D} \implies \mathbf{x}(t) \to \mathbf{x}^*
    \]
    is called the \emph{domain of attraction} of $\mathbf{x}^*$. 
\end{defn}

One sufficient condition for the above is that there exist a continuously differentiable $V \colon \mathcal{D} \to [0,\infty)$ such that
\[
    \lim_{\mathbf{x} \to \partial \mathcal{D}} V(\mathbf{x}) = \infty, \text{ and}
\]
\[
    \left\langle \nabla V(\mathbf{x}), h(\mathbf{x})\right\rangle < 0 \quad \forall \mathbf{x} \in \mathcal{D}, \mathbf{x} \neq \mathbf{x}^*.
\]
Thus, we have
\[
    \frac{\D}{\D t}V(\mathbf{x}(t)) = \left\langle \nabla V(\mathbf{x}(t)), h(\mathbf{x}(t))\right\rangle < 0 \quad \text{when }\mathbf{x}(t) \neq \mathbf{x}^*,
\]
that is, $V$ decreases along the trajectory. Since $V \geq 0$, $\mathbf{x}(t) \to \mathbf{x}^*$. Further, if we consider $\mathcal{B}_c(\mathbf{x}^*) \vcentcolon= \{ \mathbf{x} \mid V(\mathbf{x}) \leq c \} \subseteq \mathcal{D}$ for a suitable $c > V(\mathbf{x}^*$, then 
\[
    \mathbf{x}(0) \in \mathcal{B}_c(\mathbf{x}^*) \implies \mathbf{x}(t) \in \mathcal{B}_c(\mathbf{x}^*) \quad \forall t \geq 0.
\]
Note that $\mathbf{x}^* \in \mathcal{B}_c(\mathbf{x}^*)$ and $\mathcal{B}_c(\mathbf{x}^*)$ shrinks to $\{\mathbf{x}^*\}$ as $c \downarrow 0$. In particular, any $\epsilon$-neighbourhood of $\mathbf{x}^*$ contains $\mathcal{B}_c(\mathbf{x}^*)$ for sufficiently small $c$. Thus, $\mathbf{x}^*$ is Liapunov stable and hence asymptotically stable. In this case, $V$ is called a \emph{Liapunov function}. Conversely, if $\mathbf{x}^*$ is asymptotically stable, then such a $V$ exists and can be taken to satisfy $V(\mathbf{x}) \to \infty$ as $\mathbf{x} \to \partial \mathcal{D}$.

More generally, we have the \emph{LaSalle Invariance Prinsciple} which states that $\mathbf{x}(t)$ converges to the largest invariant set contained in $\mathcal{A} \vcentcolon= \{\mathbf{x} \mid \left\langle V(\mathbf{x}, h(\mathbf{x}) \right\rangle = 0 \}.$

If there exists some continuously differentiable $V \colon \mathcal{D} \to [0,\infty)$ such that
\[
    \lim_{\mathbf{x} \to \partial \mathcal{D}} V(\mathbf{x}) = \infty, \text{ and}
\]
\[
    \left\langle \nabla V(\mathbf{x}), h(\mathbf{x})\right\rangle < 0 \quad \forall \mathbf{x} \notin \mathcal{C}
\]
for some bounded set $\mathcal{C}$, then
\begin{align*}
    &\frac{\D}{\D t}V(\mathbf{x}(t)) = \left\langle \nabla V(\mathbf{x}(t)), h(\mathbf{x}(t))\right\rangle < 0 \quad \text{when }\mathbf{x}(t) \notin \mathcal{C}, \\
    &\implies \mathbf{x}(t) \to \mathcal{C}.
\end{align*}
In particular, the trajectories remain bounded. 

\medskip

We now consider the linear system $\dot{\mathbf{x}}(t) = \mathbf{Ax}(t)$ for some $\mathbf{A} \in \mathbb{R}^{d \times d}$. Then the origin, $\mathbf{0}$, is an equilibrium. If $\mathbf{A}$ is non-singular, it is the only equilibrium. It is asymptotically stable if all eigenvalues of $\mathbf{A}$ are in the left half plane. If not, suppose there are no eigenvalues on the imaginary axis. Suppose there are $m < d$ eigenvalues in the left half plane. We can write $\mathbb{R}^d = \mathcal{S} \oplus \mathcal{U}$ where $\mathcal{S}$ is the $m$-dimensional stable subspace corresponding to the eigenvectors of eigenvalues in the left half plane, and $\mathcal{U}$ is the $(d-m)$-dimensional stable subspace corresponding to the eigenvectors of eigenvalues in the right half plane. Then, $\mathbf{x}(0) \in \mathcal{S} \implies \mathbf{x}(t) \to \mathbf{0}$ and $\mathbf{x}(0) \in \mathcal{U}$ implies that $\mathbf{x}(t)$ moves away from $\mathbf{0}$. More importantly, if $\mathbf{x}(0) \notin \mathcal{S}$, $\mathbf{x}(t)$ eventually moves away from $\mathbf{0}$. That is,
\[
    \mathbf{x}(t) \to \mathbf{0} \iff \mathbf{x}(0) \in \mathcal{S}.
\]  
However, $\mathcal{S}$ has zero volume in $\mathbb{R}^d$, and thus, for a typical initial condition, $\mathbf{x}(t)$ eventually moves away from $\mathbf{0}$. The above arguments extend to any point $\mathbf{x}^* \in\mathbb{R}^d$ if we replace the linear ODE by the following affine ODE:
\[
    \dot{\mathbf{x}}(t) = \mathbf{A}\left( \mathbf{x}(t) - \mathbf{x}^* \right).
\]

\medskip

We now extend these ideas to the non-linear case. Suppose $h$ is continuously differentiable and let $Dh(\mathbf{x})$ denote its Jacobian matrix at $\mathbf{x}$, that is, the $(i,j)$ entry of of $Dh(\mathbf{x})$ is $\frac{\partial h_i}{\partial x_j}(\mathbf{x})$. By Taylor formula, for $\mathbf{x} \approx \mathbf{x}^*$, we have
\[
    h(\mathbf{x}) \approx h(\mathbf{x}^*) + Dh(\mathbf{x}^*)(\mathbf{x} - \mathbf{x}^*) = Dh(\mathbf{x}^*)(\mathbf{x} - \mathbf{x}^*).
\]
We now consider the affine ODE
\[
    \dot{\mathbf{z}}(t) = Dh(\mathbf{x}^*) \left( \mathbf{z}(t) - \mathbf{x}^* \right)
\]  
which is called the \emph{linearisation} of the original ODE at $\mathbf{x}^*$.


\begin{thm}[Hartman-Gro{\ss}man Theorem]
    Let $\mathbf{x}^*$ be a \emph{hyperbolic} equilibrium, that is, $Dh(\mathbf{x}^*)$ has no eigenvalues on the imaginary axis. Then, there exist open neighbourhoods $\mathcal{O}_1, \mathcal{O}_2$ of $\mathbf{x}^*$ such that the phase portrait of the original ODE in $\mathcal{O}_1$ and its linearisation in $\mathcal{O}_2$ can be mapped to each other by a continuous and continuously invertible transformation.
\end{thm}
Thus, `stable subspaces' morph into `stable manifolds' and `unstable subspaces' morph into `unstable manifolds'.
\section{Convergence Analysis}

Recall that our iteration is
\[
    \mathbf{x}(n+1) \vcentcolon= \mathbf{x}(n) + a(n) \left[ h(\mathbf{x}(n)) + M(n+1) \right], \quad n \geq 0.
\]
Since we view $a(n)$ as a discrete time step, we define the \emph{algorithmic time scale} as
\[
    t_0 \vcentcolon= 0, \quad t_n \vcentcolon= \sum_{m=0}^{n-1} a(m), \quad n \geq 0.
\]
Now, we define $\overline{\mathbf{x}}(t)$, $t \in [0,\infty)$ as follows. 
\begin{align*}
    \overline{\mathbf{x}}(t_n) &\vcentcolon= \mathbf{x}(n) \quad \forall n \geq 0 \text{ and} \\
    \overline{\mathbf{x}}(t) &\vcentcolon= \overline{\mathbf{x}}(t_n) + \left( \frac{t - t_n}{t_{n+1} - t_n} \right) \left( \overline{\mathbf{x}}(t_{n+1}) - \overline{\mathbf{x}}(t_n) \right) \quad \text{for } t \in [t_n, t_{n+1}].
\end{align*}
That is, we linearly interpolate on $[t_n, t_{n+1}]$. Then, $\overline{\mathbf{x}}$ is continuous and piecewise linear. 

\end{document}

